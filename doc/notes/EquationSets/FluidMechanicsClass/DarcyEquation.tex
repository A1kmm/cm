\subsection{Darcy Equation}

Darcy's equation describes the flow of fluid through a porous material. Ignoring inertial
and body forces, Darcy's equation is:

\begin{equation}
\vect{v} = -\frac{\tensor{K}}{\mu} \gradient p
\end{equation}

$\vect{v}$ is the relative fluid flow vector, given by $n (\vect{v}^f - \vect{v}^s)$ where
$n$ is the porosity, and $\vect{v}^f$ and $\vect{v}^s$ are the
Eulerian fluid and solid component velocities.
$\tensor{K}$ is the permeability tensor, $\mu$ is viscosity
and $p$ is the fluid pressure.

Conservation of mass also gives:

\begin{equation}
\gradient \cdot \vect{v} = 0
\end{equation}

The weighted residual forms of these equations over a region $\Omega$ with surface $\Gamma$ are:

\begin{equation}
\gint{\Omega}{}{\left(\vect{v} \vect{w} + \frac{\tensor{K}}{\mu} \gradient p \vect{w}\right)}{\Omega} = 0
\end{equation}

\begin{equation}
\gint{\Omega}{}{q \gradient \cdot \vect{v}}{\Omega} = 0
\end{equation}

Where $\vect{w}$ is the weighting function for the flow and $q$ is the weighting function
for pressure.
Applying Green's theorem to the pressure term to give the weak form of the equations, and multiplying through by
$\mu \tensor{K}^{-1}$:

\begin{equation}
  \gint{\Omega}{}{\left(\mu \tensor{K}^{-1} \vect{v} \vect{w} - p \gradient \vect{w}\right)}{\Omega}  + %
  \gint{\Gamma}{}{p \vect{n} \vect{w}}{\Gamma}= 0
\end{equation}

\begin{equation}
  \gint{\Omega}{}{q \gradient \cdot \vect{v}}{\Omega}= 0
\end{equation}

Where $\vect{n}$ is the normal vector to the surface $\Gamma$.

The OpenCMISS implementation of Darcy's equation uses the stabilised form of the finite element
equations developed by \cite{masud:2002}. This adds stabilising terms, so that the final form
of the implemented equations is:

\begin{equation}
  \gint{\Omega}{}{\left(\mu \tensor{K}^{-1} \vect{v} \vect{w} - p \gradient \vect{w} - %
  \frac{1}{2} \left( \gradient p \vect{w} + \mu \tensor{K}^{-1} \vect{v} \vect{w} \right) \right)}{\Omega} + %
  \gint{\Gamma}{}{p \vect{n} \vect{w}}{\Gamma}= 0
\end{equation}

\begin{equation}
  \gint{\Omega}{}{q \gradient \cdot \vect{v} + %
  \frac{1}{2} \left( \gradient q \cdot \vect{v} + %
  \frac{\tensor{K}}{\mu} \gradient p \cdot \gradient q \right)}{\Omega}= 0
\end{equation}
