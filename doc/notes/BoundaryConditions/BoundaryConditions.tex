\section{Theory}

\subsection{Dirichlet Boundary Condition}

\subsubsection{Introduction to Dirichlet boundary condition}

Dirichlet boundary condition is a type of essential boundary condition where the value of the variable in an equation 
does not change with time - for e.g. pressure or temperature at a particular computational node point. 
In contrast, a natural boundary condition can be flux in diffusion problem or stress in linear elasticity. 
Dirichlet boundary condition is enforced as a fixed specification of values at a set of boundary nodes. 
This type of boundary condition is implemented by modifying the set of equations as opposed to the right hand side of the equation.
 In a finite element framework, by enforcing the Dirichlet boundary condition an equation matrix which has been formed 
by the test function $w$ is replaced, to give a new set of equations.

The Dirichlet boundary condition being imposed on a particular nodal value implies that the specific row 
of the equation matrix corresponding to that particular node is independent of the rest of the nodes and hence can
 be eliminated and the matrix simplified in the process. For example, if the weak form of any of the problems is,



\[ \left( \begin{array}{cccc}
M_{11} & M_{12} & \hdots& M_{1N} \\
M_{21} & M_{22} & \hdots& M_{2N} \\
M_{31} & M_{32} & \hdots& M_{3N} \\
 \vdots& \vdots& \ddots& \vdots\\
M_{n1} & M_{n2} & \hdots & M_{NN} \end{array} \right) \left( \begin{array}{c}
\phi_{1} \\
\phi_{2} \\
\phi_{3} \\
 \vdots\\ 
\phi_{n} \end{array} \right) = \left( \begin{array}{c}
b_{1} \\
b_{2} \\
b_{3} \\
 \vdots\\
b_{n} \end{array} \right) \] 



Now, if you would like to impose a Dirichlet boundary condition on the first node only i.e., on $\phi_1$ as $C$, for examples as in the 
Laplace equation where Dirichlet boundary condition is implemented on the first and last node only,  


\[ \left( \begin{array}{cccc}
1 & 0 & \hdots& 0
 \end{array} \right) \left( \begin{array}{c}
\phi_{1} \\
\phi_{2} \\
\phi_{3} \\
 \vdots\\ 
\phi_{n} \end{array} \right) = \left( \begin{array}{c}
C \\
b_{2} \\
b_{3} \\
 \vdots\\
b_{n} \end{array} \right) \] 

which modifies the equation matrix as,

\[ \left( \begin{array}{cccc}
1 & 0 & \hdots& 0 \\
M_{21} & M_{22} & \hdots& M_{2N} \\
M_{31} & M_{32} & \hdots& M_{3N} \\
\vdots& \vdots& \ddots& \vdots\\
M_{n1} & M_{n2} & \hdots & M_{NN} \end{array} \right) \left( \begin{array}{c}
\phi_{1} \\
\phi_{2} \\
\phi_{3} \\
 \vdots\\ 
\phi_{n} \end{array} \right) = \left( \begin{array}{c}
C \\
b_{2} \\
b_{3} \\
 \vdots\\
b_{n} \end{array} \right) \] 


We can perform another elimination and rewrite the equation as,  

\[ \left( \begin{array}{cccc}
1 & 0 & \hdots& 0 \\
0 & M_{22} & \hdots& M_{2N} \\
0 & M_{32} & \hdots& M_{3N} \\
 \vdots& \vdots& \ddots& \vdots\\
0 & M_{n2} & \hdots & M_{NN} \end{array} \right) \left( \begin{array}{c}
\phi_{1} \\
\phi_{2} \\
\phi_{3} \\
 \vdots\\ 
\phi_{n} \end{array} \right) = \left( \begin{array}{c}
C \\
b_{2} - C M_{21} \\
b_{3} - C M_{31}\\
 \vdots\\
b_{n} - C M_{N1} \end{array} \right) \] 

Now we can store a reduced matrix as,

\[ \left( \begin{array}{cccc}
M_{22} & M_{23} & \hdots& M_{2N} \\
M_{32} & M_{33} & \hdots& M_{3N} \\
 \vdots& \vdots& \ddots& \vdots\\
M_{n2} & M_{n3} & \hdots & M_{NN} \end{array} \right)
\] 
and the right hand side is now, 

\[\left( \begin{array}{c}
C \\
(b_{2} - C M_{21}) \\
(b_{3} - C M_{31})\\
 \vdots\\
(b_{n} - C M_{N1}) \end{array} \right)
\] 
Therefore to calculate the right hand side of the equation numerically it is required to store the degrees of freedom associated 
with the node on which Dirichlet condition is enforced. The implementation of this step is done in OpenCMISS in the  \emph{boundary conditions routines}
 and is explained in detail in the following section. 

\subsubsection{Dirichlet boundary condition in OpenCMISS}

In this section we give a short description of some of the routines and OpenCMISS coding details required to modify a part of the code related 
to boundary conditions.

In an example file, for example Laplace, the boundary condition is implemented in the following steps,

\begin{itemize}
 \item  Initializing an object \emph{CmissBoundaryconditiontype}.

 \item \begin{verbatim}
     EQUATIONS_SET_BOUNDARY_CONDITIONS_CREATE_START 
    \end{verbatim} 

 - As an overview this call stack passes through the \emph{EQUATIONS SET SETUP} (for e.g. Classical field $\rightarrow$ Laplace
 $\rightarrow$ Standard Laplace setup) to use the information of the pointer to the particular `equation set' and return a pointer to the created boundary condition. 

 \item In general in the next calls we evaluated on which nodes the boundary condition is to be set. The routine 
\begin{itemize}
 \item \begin{verbatim}
 DECOMPOSITION_NODE_DOMAIN_GET
\end{verbatim} - implements a tree search to get a nodal value
 \item \begin{verbatim}
 BOUNDARY_CONDITIONS_SET_NODE
\end{verbatim} - specifies the type of boundary condition on the particular node.

\end{itemize}
 
\item \begin{verbatim}
     EQUATIONS_SET_BOUNDARY_CONDITIONS_CREATE_FINISH
    \end{verbatim} 
    \begin{itemize}
     \item  \begin{verbatim}
     EQUATIONS_SET_SETUP
    \end{verbatim} - sets up the specific details of the equation set by using the  \begin{verbatim}
     EQUATIONS_SET_SETUP_INFO
     \end{verbatim}
    \item  \begin{verbatim}
     EQUATIONS_SET_BOUNDARY_CONDITIONS_GET
    \end{verbatim} - Obtains boundary condition information for a particular type of equation set (Classical field $\rightarrow$ Laplace)
    \item \begin{verbatim}
     EQUATIONS_SET_BOUNDARY_CONDITIONS_CREATE_FINISH
    \end{verbatim} 
In \emph{boundary conditions routine}
 the Dirichlet boundary condition is imposed by first calculating the Dirichlet degrees of freedom. 
Then the row-column information of the equations matrix is obtained by obtaining information about the 
\begin{itemize}
 \item \emph{DISTRIBUTED MATRIX STORAGE TYPE}  
 \item \emph{DISTRIBUTED MATRIX STORAGE LOCATIONS}

\end{itemize}

\end{itemize}

\end{itemize}
 

 In OpenCMISS the sparse distributed matrix is stored in a format called Compressed Row Storage (CRS) format 
(more details given in the next section). There are several other possbiilities such as a column based approach: a Compressed Column Storage (CCS) format or a Block Storage type which have not been implemented yet. 

In our current structure the equation matrix is stored in the row based format i.e., CRS. 
To implement a Dirichlet boundary condition however, the information of the degrees of freedom are required which are stored in the columns of the matrix. 
Therefore it is useful to store the matrix in a column based approach in order to avoid redundant looping over all rows.
Currently a linked list based approach has been proposed which stores the matrix in a CRS format as well as a linked list format 
when a Dirichlet boundary condition is imposed on the problem. In this routine the column-row information of the matrix are obtained 
from a linked list and the degree of freedom are calculated and stored to be used in the calculation of the right 
hand side of the equation.



\subsubsection{More on matrix storage}


The Compressed row storage (CRS) is a storage format which stores the nonzero elements of a matrix sequentially. 
The storage algorithm is outlined by the following example. For a matrix of the form,

\[ \left( \begin{array}{cccccc}
 7 & 0 &-3 & 0 &-1 & 0 \\
 2 & 8 & 0 & 0 & 0 & 0 \\
 0 & 0 & 1 & 0 & 0 & 0 \\
-3 & 0 & 0 & 5 & 0 & 0 \\
 0 &-1 & 0 & 0 & 4 & 0 \\
 0 & 0 & 0 &-2 & 0 & 6 \end{array} \right)\] 

The compressed row storage or CRS is given by a pointer to the first element of row $i$ of A and a one dimensional array of the column indices,

\[ \left( \begin{array}{cccccccccccccc}
 row \hspace{0.8mm} indices \rightarrow & 1 & 4 & 7 & 9 & 11&   &  &   &   &   &   &  \\
 column \hspace{0.8mm} indices \rightarrow & 1 & 3 & 5 & 1 & 2 & 3 & 1& 4 & 2 & 5 & 4 & 6 \\
 values \rightarrow & 7 &-3 &-1 & 2 & 8 & 1 & -3 & 5 & -1 & 4 & -2& 6 \end{array} \right)\]
 
 
For a row $i$, the number of non-zero elements is easily obtained from $row \hspace{0.8mm} indices(i+1)$-$row \hspace{0.8mm} indices(i)$. For more example on the implementation
 refer to the \textit{matrix vector routines} in OpenCMISS repository.

\subsection{Neuman Boundary Condition}

\subsection{Robin Boundary Condition}