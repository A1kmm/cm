\clearemptydoublepage
\chapter{Theory}
\label{cha:theory}

\section{Equation set types}

\subsection{Static Equations}

The general form for static equations is

\subsection{Transient Equations}

The general form for transient equations is
\begin{equation}
  \matr{M}\fnof{\ddot{\vect{u}}}{t}+\matr{C}\fnof{\dot{\vect{u}}}{t}+\matr{K}\fnof{\vect{u}}{t}+
  \fnof{\vect{g}}{\fnof{\vect{u}}{t}}+\fnof{\vect{f}}{t}=\vect{0}
  \label{eqn:generaltransientnonlinear}
\end{equation}
where $\fnof{\vect{u}}{t}$ is the unknown ``displacement vector'', $\matr{M}$
is the mass matrix, $\matr{C}$ is the damping matrix, $\matr{K}$ is the
stiffness matrix, $\fnof{\vect{g}}{\fnof{\vect{u}}{t}}$ a non-linear vector
function and $\fnof{\vect{f}}{t}$ the forcing vector.

From \cite{zienkiewicz:2006_1} we now expand the unknown vector $\fnof{\vect{u}}{t}$ in terms of a polynomial of degree
$p$. With the known values of $\vect{u}_{n}$, $\dot{\vect{u}}_{n}$,
$\ddot{\vect{u}}_{n}$ up to $\symover{p-1}{\vect{u}}_{n}$ at the beginning of
the time step $\Delta t$ we can write the polynomial expansion as
\begin{equation}
  \fnof{\vect{u}}{t_{n}+\tau}\approx\fnof{\tilde{\vect{u}}}{t_{n}+\tau}=\vect{u}_{n}+\tau\dot{\vect{u}}_{n}+
  \frac{1}{2!}\tau^{2}\ddot{\vect{u}}_{n}+\cdots+\dfrac{1}{\factorial{p-1}}\tau^{p-1}\symover{p-1}{\vect{u}}_{n}+
  \dfrac{1}{p!}\tau^{p}\vect{\alpha}^{p}_{n}
  \label{eqn:timepolyexpansion}
\end{equation}
where the only unknown is the the vector $\vect{\alpha}^{p}_{n}$,
\begin{equation}
  \vect{\alpha}^{p}_{n}\approx\symover{p}{\vect{u}}\equiv\dnby{p}{\vect{u}}{t}
\end{equation}

A recurrance relationship can be established by substituting
\eqnref{eqn:timepolyexpansion} into \eqnref{eqn:generaltransientnonlinear} and
taking a weighted residual approach \ie
\begin{multline}
  \dintl{0}{\Delta
    t}\fnof{W}{\tau}\left[\matr{M}\pbrac{\ddot{\vect{u}}_{n}+\tau\dddot{\vect{u}}_{n}+\cdots+
    \dfrac{1}{\factorial{p-2}}\tau^{p-2}\vect{\alpha}^{p}_{n}} \right.\\
  +\matr{C}\pbrac{\dot{\vect{u}}_{n}+\tau\ddot{\vect{u}}_{n}+\cdots+
    \dfrac{1}{\factorial{p-1}}\tau^{p-1}\vect{\alpha}^{p}_{n}} \\
  +\matr{K}\pbrac{\vect{u}_{n}+\tau\dot{\vect{u}}_{n}+\cdots+
    \dfrac{1}{p!}\tau^{p}\vect{\alpha}^{p}_{n}} \\
  +\left.\fnof{\vect{g}}{\vect{u}_{n}+\tau\dot{\vect{u}}_{n}+\cdots+
    \dfrac{1}{p!}\tau^{p}\vect{\alpha}^{p}_{n}}+\fnof{\vect{f}}{t_{n}+\tau}\right] d\tau = \vect{0}
\end{multline}
where $\fnof{W}{\tau}$ is some weight function, $\tau=t-t_{n}$ and $\Delta
t=t_{n+1}-t_{n}$. Dividing by $\gint{0}{\Delta t}{\fnof{W}{\tau}}{\tau}$ we obtain
\begin{multline}
  \dfrac{\gint{0}{\Delta t}{\fnof{W}{\tau}\matr{M}\pbrac{\ddot{\vect{u}}_{n}+\tau\dddot{\vect{u}}_{n}+\cdots+
        \dfrac{1}{\factorial{p-2}}\tau^{p-2}\vect{\alpha}^{p}_{n}}}{\tau}}{\gint{0}{\Delta
      t}{\fnof{W}{\tau}}{\tau}} \\
  + \dfrac{\gint{0}{\Delta t}{\fnof{W}{\tau}\matr{C}\pbrac{\dot{\vect{u}}_{n}+\tau\ddot{\vect{u}}_{n}+\cdots+
        \dfrac{1}{\factorial{p-1}}\tau^{p-1}\vect{\alpha}^{p}_{n}}}{\tau}}{\gint{0}{\Delta
      t}{\fnof{W}{\tau}}{\tau}} \\
  + \dfrac{\gint{0}{\Delta t}{\fnof{W}{\tau}\matr{K}\pbrac{\vect{u}_{n}+\tau\dot{\vect{u}}_{n}+\cdots+
        \dfrac{1}{p!}\tau^{p}\vect{\alpha}^{p}_{n}}}{\tau}}{\gint{0}{\Delta
      t}{\fnof{W}{\tau}}{\tau}} \\
  + \dfrac{\gint{0}{\Delta t}{\fnof{W}{\tau}\fnof{\vect{g}}{\vect{u}_{n}+\tau\dot{\vect{u}}_{n}+\cdots+
        \dfrac{1}{p!}\tau^{p}\vect{\alpha}^{p}_{n}}}{\tau}}{\gint{0}{\Delta
      t}{\fnof{W}{\tau}}{\tau}}  
  + \dfrac{\gint{0}{\Delta t}{\fnof{W}{\tau}\fnof{\vect{f}}{t_{n}+
        \tau}}{\tau}}{\gint{0}{\Delta t}{\fnof{W}{\tau}}{\tau}}=\vect{0}
\end{multline}

Now if 
\begin{equation}
  \theta_{k}=\dfrac{\gint{0}{\Delta t}{\fnof{W}{\tau}\tau^{k}}{\tau}}{{\Delta
      t}^{k}\gint{0}{\Delta t}{\fnof{W}{\tau}}{\tau}} \text{  for  } k=0,1,\ldots,p
\end{equation}
and
\begin{equation}
  \bar{\vect{f}}=\dfrac{\gint{0}{\Delta
      t}{\fnof{W}{\tau}\fnof{\vect{f}}{t_{n}+\tau}}{\tau}}{
    \gint{0}{\Delta t}{\fnof{W}{\tau}}{\tau}}
  \label{eqn:meanweightedloadvector}
\end{equation}
we can write
\begin{multline}
  \matr{M}\pbrac{\ddot{\bar{\vect{u}}}_{n+1}+\dfrac{\theta_{p-2}{\Delta
        t}^{p-2}}{\factorial{p-2}}\vect{\alpha}^{p}_{n}}+
  \matr{C}\pbrac{\dot{\bar{\vect{u}}}_{n+1}+\dfrac{\theta_{p-1}{\Delta
        t}^{p-1}}{\factorial{p-1}}\vect{\alpha}^{p}_{n}}+
  \matr{K}\pbrac{\bar{\vect{u}}_{n+1}+\dfrac{\theta_{p}{\Delta
        t}^{p}}{p!}\vect{\alpha}^{p}_{n}}+ \\
  + \dfrac{\gint{0}{\Delta t}{\fnof{W}{\tau}\fnof{\vect{g}}{\vect{u}_{n}+\tau\dot{\vect{u}}_{n}+\cdots+
        \dfrac{1}{p!}\tau^{p}\vect{\alpha}^{p}_{n}}}{\tau}}{\gint{0}{\Delta
      t}{\fnof{W}{\tau}}{\tau}}+\bar{\vect{f}}=\vect{0}
  \label{eqn:dynamic1}
\end{multline}
where
\begin{equation}
  \begin{split}
    \bar{\vect{u}}_{n+1} &= \gsum{q=0}{p-1}{\dfrac{\theta_{q}{\Delta
            t}^{q}}{q!}\symover{q}{\vect{u}}_{n}} \\
    \dot{\bar{\vect{u}}}_{n+1} &= \gsum{q=1}{p-1}{\dfrac{\theta_{q-1}{\Delta
            t}^{q-1}}{\factorial{q-1}}\symover{q}{\vect{u}}_{n}} \\
    \ddot{\bar{\vect{u}}}_{n+1} &= \gsum{q=2}{p-1}{\dfrac{\theta_{q-2}{\Delta
            t}^{q-2}}{\factorial{q-2}}\symover{q}{\vect{u}}_{n}} 
  \end{split}
\end{equation}

We note that as $\fnof{\vect{g}}{\fnof{\vect{u}}{t}}$ is nonlinear we need to
evaluate an integral of the form
\begin{equation}
  \gint{0}{\Delta t}{\fnof{W}{\tau}\fnof{\vect{g}}{\fnof{\vect{u}}{t_{n}+\tau}}}{\tau}
\end{equation}

To do this we form Taylor's series expansions for
$\fnof{\vect{g}}{\fnof{\vect{u}}{t}}$ about the point $\fnof{\vect{u}}{t_{n}+\tau}$ \ie
\begin{equation}
  \fnof{\vect{g}}{\fnof{\vect{u}}{t_{n}}}=\fnof{\vect{g}}{\fnof{\vect{u}}{t_{n}+\tau}}-
  \tau\delby{\fnof{\vect{g}}{\fnof{\vect{u}}{t}}}{\vect{u}}\evalat{\delby{\fnof{\vect{u}}{t}}{t}}{t_{n}+\tau}
  + \orderof{\tau^{2}}
  \label{eqn:firstTaylorexpansion}
\end{equation}
and
\begin{equation}
  \fnof{\vect{g}}{\fnof{\vect{u}}{t_{n+1}}}=\fnof{\vect{g}}{\fnof{\vect{u}}{t_{n}+\tau}}+
  \pbrac{t_{n+1}-t_{n}-\tau}\delby{\fnof{\vect{g}}{\fnof{\vect{u}}{t}}}{\vect{u}}
  \evalat{\delby{\fnof{\vect{u}}{t}}{t}}{t_{n}+\tau}+ \orderof{\tau^{2}}
  \label{eqn:secondTaylorexpansion}
\end{equation}

Now if we add $\dfrac{1}{\tau}$ times \eqnref{eqn:firstTaylorexpansion} and
$\dfrac{1}{t_{n+1}-t_{n}-\tau}=\dfrac{1}{\Delta t-\tau}$ times
\eqnref{eqn:secondTaylorexpansion} we obtain
\begin{equation}
  \dfrac{\fnof{\vect{g}}{\fnof{\vect{u}}{t_{n}}}}{\tau}+\dfrac{\fnof{\vect{g}}{\fnof{\vect{u}}{t_{n+1}}}}{\Delta
    t-\tau}=\pbrac{\dfrac{\Delta t}{\tau\pbrac{\Delta t-\tau}}}\fnof{\vect{g}}{\fnof{\vect{u}}{t_{n}+\tau}}+
  \pbrac{\dfrac{\Delta t}{\tau\pbrac{\Delta t-\tau}}}\orderof{\tau^{2}}
\end{equation}

Multiplying through by $\dfrac{\tau\pbrac{\Delta t-\tau}}{\Delta t}$ gives
\begin{equation}
  \dfrac{\Delta t-\tau}{\Delta t}\fnof{\vect{g}}{\fnof{\vect{u}}{t_{n}}}+
  \dfrac{\tau}{\Delta t}\fnof{\vect{g}}{\fnof{\vect{u}}{t_{n+1}}}=
  \fnof{\vect{g}}{\fnof{\vect{u}}{t_{n}+\tau}}+\orderof{\tau^{2}}
\end{equation}

Therefore
\begin{equation}
  \dfrac{\gint{0}{\Delta t}{\fnof{W}{\tau}\fnof{\vect{g}}{\fnof{\vect{u}}{t_{n}+\tau}}}{\tau}}
  {\gint{0}{\Delta t}{\fnof{W}{\tau}}{\tau}}=\dfrac{\gint{0}{\Delta t}{\fnof{W}{\tau}
      \pbrac{\dfrac{\Delta t-\tau}{\Delta t}\fnof{\vect{g}}{\fnof{\vect{u}}{t_{n}}}+
        \dfrac{\tau}{\Delta t}\fnof{\vect{g}}{\fnof{\vect{u}}{t_{n+1}}}+\orderof{\tau^{2}}}}{\tau}}
  {\gint{0}{\Delta t}{\fnof{W}{\tau}}{\tau}}
\end{equation}

Now if we recall that
\begin{equation}
\theta_{1}=\dfrac{\gint{0}{\Delta t}{\fnof{W}{\tau}\tau}{\tau}}{\Delta t\gint{0}{\Delta t}{\fnof{W}{\tau}}{\tau}}
\end{equation}
we can write
\begin{equation}
  \dfrac{\gint{0}{\Delta t}{\fnof{W}{\tau}\fnof{\vect{g}}{\fnof{\vect{u}}{t_{n+1}}}}{\tau}}
  {\gint{0}{\Delta t}{\fnof{W}{\tau}}{\tau}}=\pbrac{1-\theta_{1}}\fnof{\vect{g}}{\fnof{\vect{u}}{t_{n}}}+
  \theta_{1}\fnof{\vect{g}}{\fnof{\vect{u}}{t_{n+1}}}+\text{Error}
\end{equation}
where
\begin{equation}
  \text{Error}=\dfrac{\gint{0}{\Delta t}{\fnof{W}{\tau}\orderof{\tau^{2}}}{\tau}}{
    \gint{0}{\Delta t}{\fnof{W}{\tau}}{\tau}}
\end{equation}

\Eqnref{eqn:dynamic1} now becomes
\begin{multline}
  \matr{M}\pbrac{\ddot{\bar{\vect{u}}}_{n+1}+\dfrac{\theta_{p-2}{\Delta
        t}^{p-2}}{\factorial{p-2}}\vect{\alpha}^{p}_{n}}+
  \matr{C}\pbrac{\dot{\bar{\vect{u}}}_{n+1}+\dfrac{\theta_{p-1}{\Delta
        t}^{p-1}}{\factorial{p-1}}\vect{\alpha}^{p}_{n}}\\
  +\matr{K}\pbrac{\bar{\vect{u}}_{n+1}+\dfrac{\theta_{p}{\Delta
        t}^{p}}{p!}\vect{\alpha}^{p}_{n}}+ 
  \pbrac{1-\theta_{1}}\fnof{\vect{g}}{\vect{u}_{n}}+\theta_{1}\fnof{\vect{g}}{\vect{u}_{n+1}}+\bar{\vect{f}}+
  \text{Error}=\vect{0}
  \label{eqn:dynamic2}
\end{multline}
as $\fnof{\vect{u}}{t_{n}}=\vect{u}_{n}$ and
$\fnof{\vect{u}}{t_{n+1}}=\vect{u}_{n+1}=\hat{\vect{u}}_{n+1}+
\dfrac{{\Delta t}^{p}}{p!}\vect{\alpha}^{p}_{n}$ where $\hat{\vect{u}}_{n+1}$
is the \emph{predicted displacement} at the new time step and is given by
\begin{equation}
  \hat{\vect{u}}_{n+1}=\gsum{q=0}{p-1}{\dfrac{{\Delta
        t}^{q}}{q!}\symover{q}{\vect{u}}_{n}}
\end{equation}

Rearranging gives
\begin{multline}
  \fnof{\vect{\psi}}{\vect{\alpha}^{p}_{n}}=\pbrac{\dfrac{\theta_{p-2}{\Delta
        t}^{p-2}}{\factorial{p-2}}\matr{M}+\dfrac{\theta_{p-1}{\Delta
        t}^{p-1}}{\factorial{p-1}}\matr{C}+\dfrac{\theta_{p}{\Delta
        t}^{p}}{p!}\matr{K}}\vect{\alpha}^{p}_{n}+\theta_{1}\fnof{\vect{g}}{\hat{\vect{u}}_{n+1}+ 
    \dfrac{{\Delta t}^{p}}{p!}\vect{\alpha}^{p}_{n}} \\
  +\pbrac{1-\theta_{1}}\fnof{\vect{g}}{\vect{u}_{n}}+
  \pbrac{\matr{M}\ddot{\bar{\vect{u}}}_{n+1}+\matr{C}\dot{\bar{\vect{u}}}_{n+1}+\matr{K}\bar{\vect{u}}_{n+1}+
    \bar{\vect{f}}}= \vect{0}
  \label{eqn:transient}
\end{multline}
or 
\begin{equation}
\fnof{\vect{\psi}}{\vect{\alpha}^{p}_{n}}=\matr{A}\vect{\alpha}^{p}_{n}+
\theta_{1}\fnof{\vect{g}}{\hat{\vect{u}}_{n+1}+ \dfrac{{\Delta
      t}^{p}}{p!}\vect{\alpha}^{p}_{n}}+\pbrac{1-\theta_{1}}\fnof{\vect{g}}{\vect{u}_{n}}+\vect{b}= \vect{0}
\end{equation}
where $\matr{A}$ is the \emph{Amplification matrix} given by
\begin{equation}
  \matr{A}=\dfrac{\theta_{p-2}{\Delta t}^{p-2}}{\factorial{p-2}}\matr{M}+
  \dfrac{\theta_{p-1}{\Delta t}^{p-1}}{\factorial{p-1}}\matr{C}+
  \dfrac{\theta_{p}{\Delta t}^{p}}{p!}\matr{K}
\end{equation}
and $\vect{b}$ is the right hand side vector given by
\begin{equation}
  \vect{b}=\matr{M}\ddot{\bar{\vect{u}}}_{n+1}+\matr{C}\dot{\bar{\vect{u}}}_{n+1}+
  \matr{K}\bar{\vect{u}}_{n+1}+\bar{\vect{f}}
\end{equation}

If $\fnof{\vect{g}}{\vect{u}}\equiv\vect{0}$ then \eqnref{eqn:transient} is linear in
$\vect{\alpha}^{p}_{n}$ and $\vect{\alpha}^{p}_{n}$ can be found by solving
the linear equation
\begin{equation}
  \vect{\alpha}^{p}_{n} =-\inverse{\pbrac{\dfrac{\theta_{p-2}{\Delta t}^{p-2}}{\factorial{p-2}}\matr{M}+
      \dfrac{\theta_{p-1}{\Delta t}^{p-1}}{\factorial{p-1}}\matr{C}+
      \dfrac{\theta_{p}{\Delta
          t}^{p}}{p!}\matr{K}}}\pbrac{\matr{M}\ddot{\bar{\vect{u}}}_{n+1}+
    \matr{C}\dot{\bar{\vect{u}}}_{n+1}+\matr{K}\bar{\vect{u}}_{n+1}+\bar{\vect{f}}}
\end{equation}
or 
\begin{equation}
  \vect{\alpha}^{p}_{n} =-\inverse{\matr{A}}\vect{b}
\end{equation}

If $\fnof{\vect{g}}{\vect{u}}$ is not $\equiv\vect{0}$ then
\eqnref{eqn:transient} is nonlinear in $\vect{\alpha}^{p}_{n}$. To solve this
equation we use Newton's method \ie
\begin{equation}
  \begin{split}
    \text{1.  } & \fnof{\matr{J}}{\vect{\alpha}^{p}_{n(i)}}.\delta
    \vect{\alpha}^{p}_{n(i)} = 
    -\fnof{\vect{\psi}}{\vect{\alpha}^{p}_{n(i)}} \\
    \text{2.  } & \vect{\alpha}^{p}_{n(i+1)}=\vect{\alpha}^{p}_{n(i)}+\delta
    \vect{\alpha}^{p}_{n(i)}
  \end{split}
\end{equation}
where $\fnof{\matr{J}}{\vect{\alpha}^{p}_{n}}$ is the Jacobian and is given by
\begin{equation}
  \fnof{\matr{J}}{\vect{\alpha}^{p}_{n}}=\dfrac{\theta_{p-2}{\Delta t}^{p-2}}{\factorial{p-2}}\matr{M}+
  \dfrac{\theta_{p-1}{\Delta
      t}^{p-1}}{\factorial{p-1}}\matr{C}+\dfrac{\theta_{p}{\Delta t}^{p}}{p!}\matr{K}+
  \dfrac{\theta_{1}{\Delta t}^{p}}{p!}
  \delby{\fnof{\vect{g}}{\hat{\vect{u}}_{n+1}+\dfrac{{\Delta
          t}^{p}}{p!}
      \vect{\alpha}^{p}_{n}}}{\vect{\alpha}^{p}_{n}}
\end{equation}
or
\begin{equation}
  \fnof{\matr{J}}{\vect{\alpha}^{p}_{n}}=\matr{A}+\dfrac{\theta_{1}{\Delta
      t}^{p}}{p!}
  \delby{\fnof{\vect{g}}{\hat{\vect{u}}_{n+1}+\dfrac{{\Delta t}^{2}}{p!}\vect{\alpha}^{p}_{n}}}{\vect{\alpha}^{p}_{n}}
\end{equation}

Once $\vect{\alpha}^{p}_{n}$ has been obtained the values at the next time step can be obtained from
\begin{equation}
  \begin{split}
    \vect{u}_{n+1} &= \vect{u}_{n}+\Delta t
    \dot{\vect{u}}_{n}+\cdots+\dfrac{{\Delta
        t}^{p}}{p!}\vect{\alpha}^{p}_{n}=\hat{\vect{u}}_{n+1}+
    \dfrac{{\Delta t}^{p}}{p!}\vect{\alpha}^{p}_{n}\\
    \dot{\vect{u}}_{n+1} &= \dot{\vect{u}}_{n}+\Delta t
    \ddot{\vect{u}}_{n}+\cdots+\dfrac{{\Delta
        t}^{p-1}}{\factorial{p-1}}\vect{\alpha}^{p}_{n}=\dot{\hat{\vect{u}}}_{n+1}+\dfrac{{\Delta
        t}^{p-1}}{\factorial{p-1}}\vect{\alpha}^{p}_{n} \\
    &\vdots \\
    \symover{p-1}{\vect{u}}_{n+1} &= \symover{p-1}{\vect{u}}_{n}+\Delta t\vect{\alpha}^{p}_{n}
  \end{split}
\end{equation}

For algorithms in which the degree of the polynomial, $p$, is higher than the
order we require the algorithm to be initialised so that the initial velocity
or acceleration can be computed. The initial velocity or acceleration values
can be obtained by substituting the initial displacement or initial
displacement and velocity values into \eqnref{eqn:generaltransientnonlinear},
rearranging and solving. For example consider an the case of a second degree
polynomial and a first order system. Substituing the initial displacement
$\vect{u}_{0}$ into \eqnref{eqn:generaltransientnonlinear} gives
\begin{equation}
  \matr{C}\dot{\vect{u}}_{0}+\matr{K}\vect{u}_{0}+\fnof{\vect{g}}{\vect{u}_{0}}+\bar{\vect{f}}_{0}=\vect{0}
\end{equation}
and therefore an approximation to the initial velocity can be found from
\begin{equation}
  \dot{\vect{u}}_{0}=-\inverse{\matr{C}}\pbrac{\matr{K}\vect{u}_{0}+\fnof{\vect{g}}{\vect{u}_{0}}+\bar{\vect{f}}_{0}}
\end{equation}

Similarily for a third degree polynomial and a second order system the initial
acceleration can be found from
\begin{equation}
  \ddot{\vect{u}}_{0}=-\inverse{\matr{M}}\pbrac{\matr{C}\dot{\vect{u}}_{0}+\matr{K}\vect{u}_{0}+
    \fnof{\vect{g}}{\vect{u}_{0}}+\bar{\vect{f}}_{0}}
\end{equation}

To evaluate the mean weighted load vector, $\bar{\vect{f}}$, we need to
evaluate the integral in \eqnref{eqn:meanweightedloadvector}. In some cases,
however, we can make the assumption that the load vector varies linearly
during the time step. In these cases the mean weighted load vector can be
computed from
\begin{equation}
  \bar{\vect{f}}=\theta_{1}\vect{f}_{n+1}+\pbrac{1-\theta_{1}}\vect{f}_{n}
\end{equation}

\subsubsection{Special SN11 case, p=1}

For this special case, the mean predicited values are given by
\begin{equation}
   \bar{\vect{u}}_{n+1} = \vect{u}_{n}
\end{equation}

The predicted displacement values are given by
\begin{equation}
   \hat{\vect{u}}_{n+1} = \vect{u}_{n}
\end{equation}

The amplification matrix is given by
\begin{equation}
  \matr{A}=\matr{C}+\theta_{1}\Delta t \matr{K}
\end{equation}

The right hand side vector is given by
\begin{equation}
  \vect{b}=\matr{K}\bar{\vect{u}}_{n+1}+\bar{\vect{f}}
\end{equation}

The nonlinear function is given by
\begin{equation}
  \fnof{\vect{\psi}}{\vect{\alpha}^{1}_{n}}=\matr{A}\vect{\alpha}^{1}_{n}+\theta_{1}\fnof{\vect{g}}{\hat{\vect{u}}_{n+1}+ 
    \Delta t\vect{\alpha}^{1}_{n}}+\pbrac{1-\theta_{1}}\fnof{\vect{g}}{\vect{u}_{n}}+\vect{b}=\vect{0}
\end{equation}

The Jacobian matrix is given by
\begin{equation}
  \fnof{\matr{J}}{\vect{\alpha}^{1}_{n}}=\matr{A}+\theta_{1}\Delta t
  \delby{\fnof{\vect{g}}{\hat{\vect{u}}_{n+1}+\Delta t\vect{\alpha}^{1}_{n}}}{\vect{\alpha}^{1}_{n}}
\end{equation}

And the time step update is given by
\begin{equation}
    \vect{u}_{n+1} = \vect{u}_{n}+\Delta t\vect{\alpha}^{1}_{n}
\end{equation}

\subsubsection{Special SN21 case, p=2}

For this special case, the mean predicited values are given by
\begin{equation}
  \begin{split}
    \bar{\vect{u}}_{n+1} &= \vect{u}_{n}+\theta_{1}\Delta t\dot{\vect{u}}_{n}\\
    \dot{\bar{\vect{u}}}_{n+1} &= \dot{\vect{u}}_{n}
  \end{split}
\end{equation}
where
\begin{equation}
  \dot{\vect{u}}_{0}=-\inverse{\matr{C}}\pbrac{\matr{K}\vect{u}_{0}+\fnof{\vect{g}}{\vect{u}_{0}}+\bar{\vect{f}}_{0}}
\end{equation}

The predicted displacement values are given by
\begin{equation}
   \hat{\vect{u}}_{n+1} = \vect{u}_{n}+\Delta t\dot{\vect{u}}_{n}
\end{equation}

The amplification matrix is given by
\begin{equation}
  \matr{A}=\theta_{1}\Delta t\matr{C}+\dfrac{\theta_{2}{\Delta t}^{2}}{2}\matr{K}
\end{equation}

The right hand side vector is given by
\begin{equation}
  \vect{b}=\matr{C}\dot{\bar{\vect{u}}}_{n+1}+\matr{K}\bar{\vect{u}}_{n+1}+\bar{\vect{f}}
\end{equation}

The nonlinear function is given by
\begin{equation}
  \fnof{\vect{\psi}}{\vect{\alpha}^{2}_{n}}=\matr{A}\vect{\alpha}^{2}_{n}+\theta_{1}\fnof{\vect{g}}{\hat{\vect{u}}_{n+1}+
    \dfrac{{\Delta t}^{2}}{2}\vect{\alpha}^{2}_{n}}+\pbrac{1-\theta_{1}}\fnof{\vect{g}}{\vect{u}_{n}}+\vect{b}=\vect{0}
\end{equation}

The Jacobian matrix is given by
\begin{equation}
  \fnof{\matr{J}}{\vect{\alpha}^{2}_{n}}=\matr{A}+\dfrac{\theta_{1}{\Delta t}^{2}}{2}
  \delby{\fnof{\vect{g}}{\hat{\vect{u}}_{n+1}+\dfrac{{\Delta t}^{2}}{2}\vect{\alpha}^{2}_{n}}}{\vect{\alpha}^{2}_{n}}
\end{equation}

And the time step update is given by
\begin{equation}
  \begin{split}
    \vect{u}_{n+1} &= \vect{u}_{n}+\Delta t\dot{\vect{u}}_{n} +\dfrac{{\Delta t}^{2}}{2}\vect{\alpha}^{2}_{n} \\
    \dot{\vect{u}}_{n+1} &= \dot{\vect{u}}_{n}+\Delta t\vect{\alpha}^{2}_{n}
  \end{split}
\end{equation}

\subsubsection{Special SN22 case, p=2}

For this special case, the mean predicited values are given by
\begin{equation}
  \begin{split}
    \bar{\vect{u}}_{n+1} &= \vect{u}_{n}+\theta_{1}\Delta t\dot{\vect{u}}_{n}\\
    \dot{\bar{\vect{u}}}_{n+1} &= \dot{\vect{u}}_{n}
  \end{split}
\end{equation}

The predicted displacement values are given by
\begin{equation}
   \hat{\vect{u}}_{n+1} = \vect{u}_{n}+\Delta t\dot{\vect{u}}_{n}
\end{equation}

The amplification matrix is given by
\begin{equation}
  \matr{A}=\matr{M}+\theta_{1}\Delta t\matr{C}+\dfrac{\theta_{2}{\Delta t}^{2}}{2}\matr{K}
\end{equation}

The right hand side vector is given by
\begin{equation}
  \vect{b}=\matr{C}\dot{\bar{\vect{u}}}_{n+1}+\matr{K}\bar{\vect{u}}_{n+1}+\bar{\vect{f}}
\end{equation}

The nonlinear function is given by
\begin{equation}
  \fnof{\vect{\psi}}{\vect{\alpha}^{2}_{n}}=\matr{A}\vect{\alpha}^{2}_{n}+\theta_{1}\fnof{\vect{g}}{\hat{\vect{u}}_{n+1}+ 
    \dfrac{{\Delta t}^{2}}{2}\vect{\alpha}^{2}_{n}}+\pbrac{1-\theta_{1}}\fnof{\vect{g}}{\vect{u}_{n}}+\vect{b}=\vect{0}
\end{equation}

The Jacobian matrix is given by
\begin{equation}
  \fnof{\matr{J}}{\vect{\alpha}^{2}_{n}}=\matr{A}+\dfrac{\theta_{1}{\Delta t}^{2}}{2}
  \delby{\fnof{\vect{g}}{{\hat{\vect{u}}_{n+1}+\dfrac{{\Delta t}^{2}}{2}\vect{\alpha}^{2}_{n}}}}{\vect{\alpha}^{2}_{n}}
\end{equation}

And the time step update is given by
\begin{equation}
  \begin{split}
    \vect{u}_{n+1} &= \vect{u}_{n}+\Delta t\dot{\vect{u}}_{n} +\dfrac{{\Delta t}^{2}}{2}\vect{\alpha}^{2}_{n} \\
    \dot{\vect{u}}_{n+1} &= \dot{\vect{u}}_{n}+\Delta t\vect{\alpha}^{2}_{n} 
  \end{split}
\end{equation}

