\subsection{Poiseuille Flow Equations}

\subsubsection{Governing equations:}

The Poiseuille equation describes the pressure drop that occurs due to viscous
friction from laminar fluid flow over a straight pipe. Poiseuille flow can be
derived from the Navier-Stokes equationsin cylindrical coordinates
($\vect{u}=\pbrac{u_{r},u_{\theta},u_{x}}$) by assuming that the flow
is: \begin{enumerate} \setlength{\itemsep}{1pt} \setlength{\parskip}{0pt}
  \setlength{\parsep}{0pt}
\item Steady \ie $\delby{\vect{u}}{t}=0$
\item The radial and swirl components of fluid velocity are zero \ie
  $u_{r}=u_{\theta}=0$ 
\item The flow is axisymmetric \ie $\delby{\vect{u}}{\theta}=0$ and fully developed
  \ie $\delby{\vect{u}}{x}=0$.
\end{enumerate}

The Poiseuille equation is:
\begin{equation}
  \boxed{
    \Delta{p}=\dfrac{8\mu LQ}{\pi R^{4}}
  }
  \label{eqn:Poiseuille_equation}
\end{equation}
where $\Delta{p}$ is the pressure drop along the pipe, $\mu$ is the viscosity $L$
is the length of the pipe $Q$ is volumetric flow and $R$ is the radius of the
pipe. 

Rearranging \eqnref{eqn:Poiseuille_equation} gives
\begin{equation}
  Q=\dfrac{\pi R^{2}R^{2}\Delta{p}}{8\mu L}
\end{equation}

Volumetric flow is average axisymmetric flow, $\bar{u}$, multipled by the area
of the pipe $A$ \ie $Q=\bar{u}A$. \Eqnref{eqn:Poiseuille_equation} now becomes
\begin{equation}
  \bar{u}A=\dfrac{AR^{2}}{8\mu}\Delta{p}
\end{equation}
or 
\begin{equation}
  \dfrac{R^{2}}{8\mu}\gradient{p}-\bar{u}=0
\end{equation}
as
\begin{equation}
  \dfrac{\Delta p}{L}=\delby{p}{x}=\gradient{p}
\end{equation}

\subsubsection{Weak formulation:}

The weak form of the differential form of the Poiseuille equation system
consisting of \eqnref{eqn:Poiseuille_equation} and can be written as:
\begin{equation}
  \gint{\Omega}{}{\pbrac{\dfrac{R^{2}}{8\mu}\gradient{p}-\bar{u}}w}{\Omega}=0
  \label{eqn:Poiseuilleweakform}
\end{equation}

\subsubsection{Finite element formulation:}
We can now discretise the domain into finite elements. \Eqnref{eqn:Poiseuilleweakform} becomes
\begin{equation}
  \dsum_{e=1}^{E}\gint{\Omega_{e}}{}{\pbrac{\frac{R^{2}}{8\mu}\gradient{p}-\bar{u}}w}{\Omega}=0
  \label{eqn:Poiseullefemform1}
\end{equation}
or
\begin{equation}
  \dsum_{e=1}^{E}\sqbrac{\gint{\Omega_{e}}{}{\pbrac{\frac{R^{2}}{8\mu}\gradient{p}w}{\Omega}-\gint{\Omega_{e}}{}{\bar{u}}w}{\Omega}}=0
  \label{eqn:Poiseullefemform2}
\end{equation}

We can now interpolate the dependent variables using basis functions
\begin{equation}
  \begin{split}
    \fnof{p}{\xi} &= \gbfn{n}{\beta}{\xi}\nodedof{p}{n}{\beta}\gsf{n}{\beta} \\
    \fnof{\bar{u}}{\xi} &= \elementdof{u}{e}
  \end{split}
\end{equation}

Using a Galerkin finite element approach (where the weighting functions are
chosen to be the interpolating basis functions) we have 
\begin{equation}
  \fnof{w}{\xi}=\gbfn{m}{\alpha}{\xi}\gsf{m}{\alpha}
\end{equation}

\subsubsection{Spatial integration:}

Adopting the standard integration notation we can write the spatial
integration term for each element in \eqnref{eqn:Poiseullefemform2} as
\begin{multline}
  \gint{0}{1}{\frac{R^{2}}{8\mu}\delby{\gbfn{n}{\beta}{\xi}\nodedof{p}{n}{\beta}\gsf{n}{\beta}}{\xi}
    \delby{\xi}{x}\gbfn{m}{\alpha}{\xi}\gsf{m}{\alpha}\abs{\fnof{J}{\xi}}}{\xi}- \\
  \gint{0}{1}{\elementdof{u}{e}\gbfn{m}{\alpha}{\xi}\gsf{m}{\alpha}\abs{\fnof{J}{\xi}}}{\xi}=\vect{0}
  \label{eqn:Poiseullefemform1}
\end{multline}
or, taking the fixed degrees-of-freedom and scale factors outside the integrals, we have
\begin{multline}
  \frac{R^{2}}{8\mu}\gsf{m}{\alpha}\gsf{n}{\beta}\nodedof{p}{n}{\beta}\gint{0}{1}{\gbfn{m}{\alpha}{\xi}
    \delby{\gbfn{n}{\beta}{\xi}}{\xi}\delby{\xi}{x}\abs{\fnof{J}{\xi}}}{\xi}- \\
  \gsf{m}{\alpha}\elementdof{u}{e}\gint{0}{1}{\gbfn{m}{\alpha}{\xi}\abs{\fnof{J}{\xi}}}{\xi}=\vect{0}
  \label{eqn:Poiseullefemform2}
\end{multline}

This is an elemental equation of the form
\begin{equation}
  K^{\alpha\beta}_{mn}\nodedof{p}{n}{\beta}-f^{\alpha}_{m}\elementdof{u}{e}=\vect{0}
\end{equation}
or
\begin{equation}
  \left[ \begin{array}{cc}
      K^{\alpha\beta}_{mn} & -f^{\alpha}_{m} \\
      \transpose{\vect{0}} & 0
    \end{array} \right] \left[ \begin{array}{c}
      \nodedof{p}{n}{\beta} \\
      \elementdof{u}{e}
    \end{array} \right] =  \left[ \begin{array}{c}
      \vect{0} \\
      0
    \end{array} \right]
\end{equation}

The elemental stiffness matrix, $K^{\alpha\beta}_{mn}$, is given by
\begin{equation}
  K^{\alpha\beta}_{mn}=\frac{R^{2}}{8\mu}\gsf{m}{\alpha}\gsf{n}{\beta}
  \gint{0}{1}{\gbfn{m}{\alpha}{\xi}\delby{\gbfn{n}{\beta}{\xi}}{\xi}\delby{\xi}{x}
    \abs{\fnof{J}{\xi}}}{\xi}
  \label{eqn:elementalfemlhs1}
\end{equation}
and the elemental stiffness vector, $f^{\alpha}_{m}$, is given by
\begin{equation}
  f^{\alpha}_{m}=\gsf{m}{\alpha}\elementdof{u}{e}\gint{0}{1}{\gbfn{m}{\alpha}{\xi}
    \abs{\fnof{J}{\xi}}}{\xi}
  \label{eqn:elementalfemlhs2}
\end{equation}

Note that an additional conservation of mass equations are required to solve
the final matrix system.
