\clearemptydoublepage
\chapter{Modules}
\label{cha:modules}

\section{Definitions}
\label{sec:definitions}

\emph{Class}, \emph{type} and \emph{sub-type} - In \emph{OpenCMISS}, \emph{class} 
is the broadest grouping, it is a means of identifying equations of a physics 
type (for example: \emph{Fluid Mechanics}), \emph{type} refers to a particular 
equation in that \emph{class} (for example: \emph{Navier Stokes}), and 
\emph{sub-type} refers to an implementation of that equation (for example: 
\emph{static Navier Stokes}). \\

\noindent Variable, parameter and routine naming - \emph{OpenCMISS} uses descriptive variable, 
parameter and routine names where possible, in addition to this, some of the variables have 
legacy names from the old \emph{CMISS} code (for example, $nd$ for data, $ng$ for gauss, 
etc), these are in process of being replaced with descriptive names. \\
\linebreak
\emph{Hex} and \emph{Tet} - In addition to \emph{Hex} (Hexahedral) elements, 
\emph{OpenCMISS} can handle \emph{Tet} (Tetrahedral) elements, linear, quadratic 
and cubic interpolation. It should be noted that \emph{cmgui}, whilst being 
able to visualise \emph{Tet} elements, cannot handle \emph{Tet} elements with 
cubic interpolation. \\
\linebreak
\emph{dof} - Degree of freedom \\
\linebreak
\emph{Modules} - A collection of subroutines organised into a functional file. \\
\linebreak
\emph{Example} program - The top level program set up by the user, containing mostly
subroutine calls that call routines in the opencmiss module. \\
\linebreak
\emph{Units} - A functional unit is defined as the basic
call unit between the calls of \emph{MODULENAME\_CREATE\_START} to 
\emph{MODULENAME\_CREATE\_FINISH}. Further calls can be added between these calls to 
override default set up settings. \\
\linebreak
\emph{MODULENAME\_CREATE\_START} - This routine initialises the object. \\
\linebreak
\emph{MODULENAME\_CREATE\_FINISH} - This routine complete the object creation. \\
\linebreak
\emph{MODULENAME\_DESTROY} - This routine either deallocates all allocated variables on an object,
zeros parameters and nulls character strings, or calls a \emph{MODULENAME\_FINALISE} routine 
that performs these operations.


\section{Use of the cache}
\label{sec:cache}

\section{Diagnostics}
\label{sec:diagnostics}

\section{Library: \emph{CellML}}
\label{sec:cellml}

\section{Library: \emph{FieldML}}
\label{sec:fieldml}

\section{Module: \emph{base\_routines}}
\label{sec:baseroutines}

\section{Module: \emph{basis\_routines}}
\label{sec:basisroutines}

\section{Module: \emph{boundary\_condition\_routines}} 
\label{sec:boundaryconditionroutines}

\section{Module: \emph{classical\_field\_routines}}
\label{sec:classicalfieldroutines}

\section{Module: \emph{cmiss}}
\label{sec:cmiss}

\section{Module: \emph{cmiss\_petsc} \& \emph{cmiss\_parmetis}}
\label{sec:cmisspetscandcmissparmetis}

\section{Module: \emph{control\_loop\_routines}}
\label{sec:controllooproutines}

\section{Module: \emph{coordinate\_routines}}
\label{sec:coordinateroutines}

\section{Module: \emph{data\_point\_routines} \&  \\ 
\emph{data\_projection\_routines}}
\label{sec:dataprojectionroutines}

\section{Module: \emph{distributed\_matrix\_vector}}
\label{sec:distributedmatrixvector}

\section{Module: \emph{domain\_mappings}} 
\label{sec:domainmappings}

\section{Module: \emph{equations\_mapping\_routines}}
\label{sec:equationsmappingroutines}

\section{Module: \emph{equations\_matrices\_routines}}
\label{sec:equationsmatricesroutines}

\section{Module: \emph{equations\_routines}}
\label{sec:equationsroutines}

\section{Module: \emph{equations\_set\_constants}}
\label{sec:equationssetconstants}

\section{Module: \emph{equations\_set\_routines}}
\label{sec:equationssetroutines}

\section{Module: \emph{field\_routines}}
\label{sec:fieldroutines}

\section{Module: \emph{finite\_element\_routines}}
\label{sec:finiteelementroutines}

This module is not currently used in \emph{OpenCMISS}.


\section{Module: \emph{fitting\_routines}}
\label{sec:fittingroutines}


\section{Module: \emph{generated\_mesh\_routines}}
\label{sec:generatedmeshroutines}


\section{Interface modules}
\label{sec:interfacemodules}


\section{Module: \emph{kinds}}
\label{sec:kinds}

This module handles the different data types available in \emph{OpenCMISS}: 
\emph{\_DP} (double precision), \emph{\_INTG} (integer), \emph{\_SP} (single 
precision) or \emph{\_L} (logical). The \emph{\_} (a Fortran source term) 
means a different \emph{kind} in Fortran. In addition to these data types, 
a number of additional kinds are defined preceded by \emph{CMISS} these 
are equal to a corresponding kind, for example: \emph{\_CMISSDP} is 
equal to \emph{\_DP}. 


\section{Module: \emph{lists}}
\label{sec:lists}

This module contains the routines for lists, lists are data structures that 
allows the listing of items, their ordering and de-duplication.


\section{Module: \emph{matrix\_vector}}
\label{sec:matrixvector}


\section{Module: \emph{mesh\_routines}}
\label{sec:meshroutines}


\section{Module: \\ \emph{multi\_compartment\_transport\_routines}}
\label{sec:multicompartmenttransportroutines}


\section{Module: \emph{multi\_physics\_routines}}
\label{sec:multiphysicsroutines}

The routines in this module are for dealing with coupled problems.


\section{Module: \emph{node\_routines}}
\label{sec:noderoutines}

Nodes in \emph{OpenCMISS} store the nodal information. Nodes are 
associated with either an interface or a region, and have both a global 
and a user number (allowing the user to define their own numbering scheme 
to the nodes). In \emph{OpenCMISS} nodes are just labels for dofs, they 
do not have a position. The position in space of a dof is a property of 
the field. Dofs are important in \emph{OpenCMISS}, they are the basic topological 
unit, they are the basic unit by which matrices are indexed and 
calculations performed.


\section{Module: \emph{opencmiss}}
\label{sec:opencmiss}

The opencmiss module handles the bindings, the routines in this module act as 
an interface layer between the calls from the example program and the main 
\emph{OpenCMISS} library routines. The use of this layer allows the further 
extension of the code to enable other languages to call the main library 
routines (for example: \emph{C} and \emph{Python}).


\section{Module: \emph{problem\_constants}}
\label{sec:problemconstants}


\section{Module: \emph{problem\_routines}}
\label{sec:problemroutines}


\section{Module: \emph{region\_routines}}
\label{sec:regionroutines}


\section{Module: \emph{solver\_mapping\_routines}}
\label{sec:solvermappingroutines}


\section{Module: \emph{solver\_matrices\_routines}}
\label{sec:solvermatricesroutines}

These routines handle the solver matrices which are the actually matrices to 
be solved for the numerical problem. Solver matrices are used with the solvers, 
either internal (the \emph{CMISS} solver) or external (\emph{PETSc}, 
\emph{GMRES}) solvers.


\section{Module: \emph{solver\_routines}}
\label{sec:solverroutines}


\section{Module: \emph{sorting}}
\label{sec:sorting}

This module contains the routines used for sorting. A number sorting methods 
are available in \emph{OpenCMISS}: bubble, heap and shell sort. The default 
sorting method is heap sort. The sort methods sorts lists into ascending order. 


\section{Module: \emph{strings}}
\label{sec:strings}

This module contains the routines used for manipulating strings in 
\emph{OpenCMISS}.


\section{Module: \emph{trees}}
\label{sec:trees}

Binary search trees are used in \emph{OpenCMISS} to match a user-defined node 
number to a global or a local number of a node. \emph{Red-black} binary trees 
are used, as they have good order properties for lookups, searching and ordering, 
and good worst case characteristics. Trees are composed of keys and data, the 
search is made on the key and the return is whatever is required.

Nodes have a user node number assigned by the user and a corresponding 
global/local number that the program uses. User node numbers are used in the 
example file to reference the nodes in the mesh, their use allows users flexibility 
in their ordering and referencing of the nodes. Presently trees are only used for 
nodes lookup.

Trees are used in \emph{OpenCMISS} by two objects: the \emph{mesh object} uses 
trees to lookup the global node number, whilst the \emph{domain object} uses trees 
to lookup the local node number.
