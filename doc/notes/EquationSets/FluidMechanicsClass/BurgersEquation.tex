\subsection{Burgers's Equations}

\subsubsection{Generalised Burgers's Equation}

\subsubsubsection{Governing equations:}

For a given velocity $\fnof{u}{x,t}$ and a kinematic viscosity of $\nu$, Burgers's equation
in one dimension is
\begin{equation}
  \delby{u}{t}+u\delby{u}{x}=\nu\deltwosqby{u}{x}
  \label{eqn:Burgersequation}
\end{equation}

The general form of this equation in \OpenCMISS is
\begin{equation}
  \fnof{a}{\vect{x}}\delby{\fnof{\vect{u}}{\vect{x},t}}{t}+\fnof{b}{\vect{x}}\laplacian{\fnof{\vect{u}}{\vect{x},t}}+
  \fnof{c}{\vect{x}}\dotprod{\fnof{\vect{u}}{\vect{x},t}}{\gradient{\fnof{\vect{u}}{\vect{x},t}}}=\vect{0}
  \label{eqn:GeneralFormBurgersequation}
\end{equation}
where $\fnof{\vect{u}}{\vect{x},t}$ is the velocity vector and
$\fnof{a}{\vect{x}}$, $\fnof{b}{\vect{x}}$ and $\fnof{c}{\vect{x}}$ are
material parameters. The standard form of Burgers's equation can be found with
$a=1$, $b=-\nu$ and $c=1$.

Appropriate boundary conditions conditions for Burgers's
equation are specification of Dirichlet boundary conditions on the solution,
$\vect{d}$ \ie
\begin{equation}
  \fnof{\vect{u}}{\vect{x},t} = \fnof{\vect{d}}{\vect{x},t} \quad \vect{x}\in\Gamma_{D},
  \label{eqn:BurgersDirichletBC} 
\end{equation}
and/or Neumann conditions in terms of the solution flux in the normal
direction, $\vect{e}$ \ie
\begin{equation}
  \fnof{\vect{q}}{\vect{x},t} = \dotprod{\pbrac{\fnof{b}{\vect{x}}
      \gradient{\fnof{\vect{u}}{\vect{x},t}}}}{\normal} =
  \fnof{\vect{e}}{\vect{x},t} \quad \vect{x}\in\Gamma_{N},
  \label{eqn:BurgersNeumannBC} 
\end{equation}
where $\fnof{\vect{q}}{\vect{x},t}$ is the flux in the normal direction, $\normal$ is the normal
vector to the boudary and $\Gamma=\union{\Gamma_{D}}{\Gamma_{N}}$.

Appropriate initial conditions for the diffusion equation are the
specification of an initial value of the solution, $\vect{f}$ \ie
\begin{equation}
  \fnof{\vect{u}}{\vect{x},0} = \fnof{\vect{f}}{\vect{x}} \quad \vect{x}\in\Omega.
  \label{eqn:BurgersInitialC} 
\end{equation}

\subsubsubsection{Weak formulation:}

The corresponding weak form of \eqnref{eqn:GeneralFormBurgersequation} can be
found by integrating over the domain with test functions \ie
\begin{equation}
  \gint{\Omega}{}{\pbrac{a\delby{\vect{u}}{t}+
      b\laplacian{\vect{u}}+
      c\dotprod{\vect{u}}{\gradient{\vect{u}}}}\vect{w}}{\Omega}=\vect{0}
  \label{eqn:Burgersweakform1}
\end{equation}
where $\vect{w}$ are suitable spatial test functions.

Applying the divergence theorem in \eqnref{eqn:DivergenceTheormScalarVector} to \eqnref{eqn:Burgersweakform1} gives
\begin{equation}
  \gint{\Omega}{}{\pbrac{a\delby{\vect{u}}{t}}\vect{w}}{\Omega}-
      \gint{\Omega}{}{b\dotprod{\gradient{\vect{u}}}{\gradient{\vect{w}}}}{\Omega}
      +\gint{\Gamma}{}{\pbrac{b\dotprod{\gradient{\vect{u}}}{\normal}}\vect{w}}{\Gamma}+
      \gint{\Omega}{}{\pbrac{c\dotprod{\vect{u}}{\gradient{\vect{u}}}}\vect{w}}{\Omega}
      =\vect{0}
  \label{eqn:Burgersweakform2}
\end{equation}

\subsubsubsection{Tensor notation:}

\Eqnref{eqn:Burgersweakform2} can be written in tensor notation as
\begin{equation}
  \gint{\Omega}{}{a\dot{u}_{i}w_{i}}{\Omega}-
  \gint{\Omega}{}{bG^{jk}\covarderiv{u_{i}}{j}\covarderiv{w_{i}}{k}}{\Omega}+
  \gint{\Gamma}{}{bG^{jk}\covarderiv{u_{i}}{k}n_{j}w_{i}}{\Gamma} +
  \gint{\Omega}{}{cG^{jk}u_{j}\covarderiv{u_{i}}{k}w_{i}}{\Omega}=\vect{0}
  \label{eqn:Burgerstensorform1}
\end{equation}
or
\begin{equation}
  \gint{\Omega}{}{a\dot{u}_{i}w_{i}}{\Omega}-
  \gint{\Omega}{}{bG^{jk}\covarderiv{u_{i}}{j}\covarderiv{w_{i}}{k}}{\Omega}+
  \gint{\Gamma}{}{q_{i}w_{i}}{\Gamma} +
  \gint{\Omega}{}{cG^{jk}u_{j}\covarderiv{u_{i}}{k}w_{i}}{\Omega}=\vect{0}
  \label{eqn:Burgerstensorform2}
\end{equation}
or
\begin{multline}
  \gint{\Omega}{}{a\dot{u}_{i}w_{i}}{\Omega}-
  \gint{\Omega}{}{bG^{jk}\pbrac{\partialderiv{u_{i}}{j}-\christoffel{i}{j}{h}u_{h}}
    \pbrac{\partialderiv{w_{i}}{k}-\christoffel{i}{k}{l}w_{l}}}{\Omega}+ \\
  \gint{\Gamma}{}{q_{i}w_{i}}{\Gamma} +
  \gint{\Omega}{}{cG^{jk}u_{j}\pbrac{\partialderiv{u_{i}}{k}-
      \christoffel{i}{k}{h}u_{h}}w_{i}}{\Omega}=\vect{0}
  \label{eqn:Burgerstensorform3}
\end{multline}
where $G^{jk}$ is the contravariant metric tensor and $\christoffel{i}{j}{k}$
is the Christoffel symbol for the spatial coordinates.

\subsubsubsection{Finite element formulation:}

We can now discretise the domain into finite elements \ie $\Omega=
\displaystyle{\bigcup_{e=1}^{E}}\Omega_{e}$ with
$\Gamma=\displaystyle{\bigcup_{f=1}^{F}}\Gamma_{f}$. \Eqnref{eqn:Burgerstensorform2}
now becomes
\begin{multline}
  \dsum_{e=1}^{E}\gint{\Omega_{e}}{}{a\dot{u}_{i}w_{i}}{\Omega}-
  \dsum_{e=1}^{E}\gint{\Omega_{e}}{}{bG^{jk}\pbrac{\partialderiv{u_{i}}{j}-
      \christoffel{i}{j}{h}u_{h}} \pbrac{\partialderiv{w_{i}}{k}-
      \christoffel{i}{k}{l}w_{l}}}{\Omega}+ \\
  \dsum_{f=1}^{F}\gint{\Gamma_{f}}{}{q_{i}w_{i}}{\Gamma} +
  \dsum_{e=1}^{E}\gint{\Omega_{e}}{}{cG^{jk}u_{j}\pbrac{\partialderiv{u_{i}}{k}-
      \christoffel{i}{k}{h}u_{h}}w_{i}}{\Omega}=\vect{0}
  \label{eqn:Burgersfemform}
\end{multline}

If we now assume that the dependent variable $\vect{u}$ can be interpolated
separately in space and in time we can write
\begin{equation}
  \fnof{\vect{u}}{\vect{x},t}=\gbfn{n}{}{\vect{x}}\fnof{\nodept{\vect{u}}{n}}{t}
\end{equation}
or, in standard interpolation notation within an element,
\begin{equation}
  \fnof{u_{i}}{\vect{\xi},t}=\idxgbfn{i}{n}{\beta}{\vect{\xi}}
  \fnof{\idxnodedof{u}{i}{n}{\beta}}{t}\idxgsf{i}{n}{\beta}
\end{equation}
where $\fnof{\idxnodedof{u}{i}{n}{\beta}}{t}$ are the time varying nodal
degrees-of-freedom for velocity component $i$, node $n$, global derivative $\beta$,
$\idxgbfn{i}{n}{\beta}{\vect{\xi}}$ are the corresponding basis functions 
and $\idxgsf{i}{n}{\beta}$ are the scale factors. 

We can also interpolate the other variables in a similar manner \ie
\begin{equation}
  \begin{split}
    \fnof{q_{i}}{\vect{\xi},t} &= \idxgbfn{i}{o}{\gamma}{\vect{\xi}}
    \fnof{\idxnodedof{q}{i}{o}{\gamma}}{t}\idxgsf{i}{o}{\gamma} \\
    \fnof{a}{\vect{\xi}} &=\gbfn{p}{\delta}{\vect{\xi}}
    \nodedof{a}{p}{\delta}\gsf{p}{\delta} \\
    \fnof{b}{\vect{\xi}} &=\gbfn{p}{\delta}{\vect{\xi}}\nodedof{b}{p}{\delta}
    \gsf{p}{\delta} \\
    \fnof{c}{\vect{\xi}}
    &=\gbfn{p}{\delta}{\vect{\xi}}\nodedof{c}{p}{\delta}\gsf{p}{\delta}
  \end{split}
\end{equation}
where $\fnof{\nodedof{q_{i}}{o}{\gamma}}{t}$, $\nodedof{a}{p}{\delta}$,
$\nodedof{b}{p}{\delta}$ and $\nodedof{c}{p}{\delta}$ are the
nodal degrees-of-freedom for the variables.

For a Galerkin finite element formulation we also choose the spatial weighting
function $\vect{w}$ to be equal to the basis fucntions \ie
\begin{equation}
  \fnof{w_{i}}{\vect{\xi}}=\idxgbfn{i}{m}{\alpha}{\vect{\xi}}\idxgsf{i}{m}{\alpha}
\end{equation}

\subsubsubsection{Spatial integration:}

Adopting the standard integration notation we can write the spatial
integration term in \eqnref{eqn:Burgersfemform} as
\begin{multline}
  \dsum_{e=1}^{E}\gint{\vect{0}}{\vect{1}}{a\delby{\idxgbfn{i}{n}{\beta}{\vect{\xi}}
      \fnof{\idxnodedof{u}{i}{n}{\beta}}{t}\idxgsf{i}{n}{\beta}}{t}
    \idxgbfn{i}{m}{\alpha}{\vect{\xi}}\idxgsf{i}{m}{\alpha}
    \abs{\fnof{\matr{J}}{\vect{\xi}}}}{\vect{\xi}}- \\
  \dsum_{e=1}^{E}\dintl{\vect{0}}{\vect{1}}bG^{jk}\pbrac{\delby{\idxgbfn{i}{n}{\beta}{\vect{\xi}}
        \fnof{\idxnodedof{u}{i}{n}{\beta}}{t}\idxgsf{i}{n}{\beta}}{x^{j}}-
      \christoffel{i}{j}{h}\idxgbfn{h}{n}{\beta}{\vect{\xi}}
      \fnof{\idxnodedof{u}{h}{n}{\beta}}{t}\idxgsf{h}{n}{\beta}} \\
    \pbrac{\delby{\idxgbfn{i}{m}{\alpha}{\vect{\xi}}\idxgsf{i}{m}{\alpha}}{x^{k}}-
      \christoffel{i}{k}{l}\idxgbfn{l}{m}{\alpha}{\vect{\xi}}
      \fnof{\idxnodedof{u}{h}{n}{\beta}}{t}
      \idxgsf{l}{m}{\alpha}}\abs{\fnof{\matr{J}}{\vect{\xi}}}d\vect{\xi} + \\
  \dsum_{f=1}^{F}\gint{\vect{0}}{\vect{1}}{\idxgbfn{i}{o}{\gamma}{\vect{\xi}}
    \fnof{\idxnodedof{q}{i}{o}{\gamma}}{t}\idxgsf{i}{o}{\gamma}
    \idxgbfn{i}{m}{\alpha}{\vect{\xi}}\idxgsf{i}{m}{\alpha}
    \abs{\fnof{\matr{J}}{\vect{\xi}}}}{\vect{\xi}} + \\
  \dsum_{e=1}^{E}\dintl{\vect{0}}{\vect{1}}cG^{jk}\idxgbfn{j}{n}{\beta}{\vect{\xi}}
  \fnof{\idxnodedof{u}{j}{n}{\beta}}{t}\idxgsf{i}{n}{\beta}
  \pbracr{\delby{\idxgbfn{i}{n}{\beta}{\vect{\xi}}\fnof{\idxnodedof{u}{i}{n}{\beta}}{t}
      \idxgsf{i}{n}{\beta}}{x^{k}}}- \\
    \pbracl{\christoffel{i}{k}{h}\idxgbfn{h}{m}{\alpha}{\vect{\xi}}
    \fnof{\idxnodedof{u}{h}{n}{\beta}}{t}\idxgsf{h}{m}{\alpha}}
  \idxgbfn{i}{m}{\alpha}{\vect{\xi}}\idxgsf{i}{m}{\alpha}
  \abs{\fnof{\matr{J}}{\vect{\xi}}}d\vect{\xi}
  \label{eqn:Burgersfemform}
\end{multline}
