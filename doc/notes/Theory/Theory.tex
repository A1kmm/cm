\clearemptydoublepage
\chapter{Theory}
\label{cha:theory}

\section{Equation set types}

\subsection{Static-linear Equations}

\subsection{Static-nonlinear Equations}

\subsection{Transient-linear Equations}

\subsection{Transient-nonlinear Equations}

The general form for transient non-linear equations is

\begin{equation}
  \matr{M}\ddot{\vect{u}}+\matr{C}\dot{\vect{u}}+\matr{K}\vect{u}+\fnof{\vect{g}}{\vect{u}}+\vect{f}=\vect{0}
  \label{eqn:generaltransientnonlinear}
\end{equation}

where $\matr{M}$ is the mass matrix, $\matr{C}$ is the damping matrix,
$\matr{K}$ is the damping matrix, $\fnof{\vect{g}}{\vect{u}}$ a non-linear vector function and $\vect{f}$ the
forcing vector.

From \cite{zienkiewicz:2006_1} we now expand the unknown vector $\vect{u}$ in terms of a polynomial of degree
$p$. With the known values of $\vect{u}_{n}$, $\dot{\vect{u}}_{n}$,
$\ddot{\vect{u}}_{n}$ up to $\symover{p-1}{\vect{u}}_{n}$ at the beginning of
the time step $\Delta t$ we can write the polynomial expansion as

\begin{equation}
  \vect{u}\approx\hat{\vect{u}}=\vect{u}_{n}+\tau\dot{\vect{u}}_{n}+\frac{1}{2!}\tau^{2}\ddot{\vect{u}}_{n}+\cdots+
  \dfrac{1}{\factorial{p-1}}\tau^{p-1}\symover{p-1}{\vect{u}}_{n}+\dfrac{1}{p!}\tau^{p}\vect{\alpha}^{p}_{n}
  \label{eqn:timepolyexpansion}
\end{equation}

where the only unknown is the the vector $\vect{\alpha}^{p}_{n}$,

\begin{equation}
  \vect{\alpha}^{p}_{n}\approx\symover{p}{\vect{u}}\equiv\dnby{p}{\vect{u}}{t}
\end{equation}

A recurrance relationship can be established by substituting
\eqnref{eqn:timepolyexpansion} into \eqnref{eqn:generaltransientnonlinear} and
taking a weighted residual approx

\begin{multline}
  \dintl{0}{\Delta
    t}\fnof{W}{\tau}\left[\matr{M}\pbrac{\ddot{\vect{u}}_{n}+\tau\dddot{\vect{u}}_{n}+\cdots+
    \dfrac{1}{\factorial{p-2}}\tau^{p-2}\vect{\alpha}^{p}_{n}} \right.\\
  +\matr{C}\pbrac{\dot{\vect{u}}_{n}+\tau\ddot{\vect{u}}_{n}+\cdots+
    \dfrac{1}{\factorial{p-1}}\tau^{p-1}\vect{\alpha}^{p}_{n}} \\
  +\matr{K}\pbrac{\vect{u}_{n}+\tau\dot{\vect{u}}_{n}+\cdots+
    \dfrac{1}{p!}\tau^{p}\vect{\alpha}^{p}_{n}} \\
  +\left.\fnof{\vect{g}}{\vect{u}_{n}+\tau\dot{\vect{u}}_{n}+\cdots+
    \dfrac{1}{p!}\tau^{p}\vect{\alpha}^{p}_{n}}+\vect{f}\right] dt = \vect{0}
\end{multline}

now if 

\begin{equation}
  \theta_{k}=\dfrac{\gint{0}{\Delta t}{\fnof{W}{\tau}\tau^{k}}{\tau}}{{\Delta
      t}^{k}\gint{0}{\Delta t}{\fnof{W}{\tau}}{\tau}} \text{  for  } k=0,1,\ldots,p
\end{equation}
and
\begin{equation}
  \bar{\vect{f}}=\dfrac{\gint{0}{\Delta t}{\fnof{W}{\tau}\vect{f}}{\tau}}{{\Delta
      t}^{k}\gint{0}{\Delta t}{\fnof{W}{\tau}}{\tau}}
\end{equation}

we can write

\begin{multline}
  \matr{M}\pbrac{\ddot{\bar{\vect{u}}}_{n+1}+\dfrac{\theta_{p-2}{\Delta
        t}^{p-2}}{\factorial{p-2}}\vect{\alpha}^{p}_{n}}+
  \matr{C}\pbrac{\dot{\bar{\vect{u}}}_{n+1}+\dfrac{\theta_{p-1}{\Delta
        t}^{p-1}}{\factorial{p-1}}\vect{\alpha}^{p}_{n}}+
  \matr{K}\pbrac{\bar{\vect{u}}_{n+1}+\dfrac{\theta_{p}{\Delta
        t}^{p}}{p!}\vect{\alpha}^{p}_{n}}+ \\
  \fnof{\vect{g}}{\bar{\vect{u}}_{n+1}+\dfrac{\theta_{p}{\Delta
        t}^{p}}{p!}\vect{\alpha}^{p}_{n}}+\bar{\vect{f}}=\vect{0}
\end{multline}

where

\begin{equation}
  \begin{split}
    \bar{\vect{u}}_{n+1} &= \gsum{q=0}{p-1}{\dfrac{\theta_{q}{\Delta
            t}^{q}}{q!}\symover{q}{\vect{u}}_{n}} \\
    \dot{\bar{\vect{u}}}_{n+1} &= \gsum{q=1}{p-1}{\dfrac{\theta_{q-1}{\Delta
            t}^{q-1}}{\factorial{q-1}}\symover{q}{\vect{u}}_{n}} \\
    \ddot{\bar{\vect{u}}}_{n+1} &= \gsum{q=2}{p-1}{\dfrac{\theta_{q-2}{\Delta
            t}^{q-2}}{\factorial{q-2}}\symover{q}{\vect{u}}_{n}} 
  \end{split}
\end{equation}

Rearranging gives

\begin{multline}
  \fnof{\vect{\psi}}{\vect{\alpha}^{p}_{n}}=\pbrac{\dfrac{\theta_{p-2}{\Delta
        t}^{p-2}}{\factorial{p-2}}\matr{M}+\dfrac{\theta_{p-1}{\Delta
        t}^{p-1}}{\factorial{p-1}}\matr{C}+\dfrac{\theta_{p}{\Delta
        t}^{p}}{p!}\matr{K}}\vect{\alpha}^{p}_{n}+ \\
  \fnof{\vect{g}}{\bar{\vect{u}}_{n+1}+\dfrac{\theta_{p}{\Delta t}^{p}}{p!}\vect{\alpha}^{p}_{n}}+ 
  \pbrac{\matr{M}\ddot{\bar{\vect{u}}}_{n+1}+\matr{C}\dot{\bar{\vect{u}}}_{n+1}+\matr{K}\bar{\vect{u}}_{n+1}+
    \bar{\vect{f}}}= \vect{0}
  \label{eqn:transient}
\end{multline}

or 

\begin{equation}
\fnof{\vect{\psi}}{\vect{\alpha}^{p}_{n}}=\matr{A}\vect{\alpha}^{p}_{n}+
\fnof{\vect{g}}{\bar{\vect{u}}_{n+1}+ \dfrac{\theta_{p}{\Delta
      t}^{p}}{p!}\vect{\alpha}^{p}_{n}}+
\pbrac{\matr{M}\ddot{\bar{\vect{u}}}_{n+1}+\matr{C}\dot{\bar{\vect{u}}}_{n+1}+
  \matr{K}\bar{\vect{u}}_{n+1}+\bar{\vect{f}}}= \vect{0}
\end{equation}

where $\matr{A}$ is the \emph{Amplification matrix} given by

\begin{equation}
  \matr{A}=\dfrac{\theta_{p-2}{\Delta t}^{p-2}}{\factorial{p-2}}\matr{M}+
  \dfrac{\theta_{p-1}{\Delta t}^{p-1}}{\factorial{p-1}}\matr{C}+
  \dfrac{\theta_{p}{\Delta t}^{p}}{p!}\matr{K}
\end{equation}

If $\fnof{\vect{g}}{\vect{u}}\equiv\vect{0}$ then \eqnref{eqn:transient} is linear in
$\vect{\alpha}^{p}_{n}$ and $\vect{\alpha}^{p}_{n}$ can be found by solving
the linear equation

\begin{equation}
  \vect{\alpha}^{p}_{n} =\inverse{\pbrac{\dfrac{\theta_{p-2}{\Delta t}^{p-2}}{\factorial{p-2}}\matr{M}+
      \dfrac{\theta_{p-1}{\Delta t}^{p-1}}{\factorial{p-1}}\matr{C}+
      \dfrac{\theta_{p}{\Delta
          t}^{p}}{p!}\matr{K}}}\pbrac{\matr{M}\ddot{\bar{\vect{u}}}_{n+1}+
    \matr{C}\dot{\bar{\vect{u}}}_{n+1}+\matr{K}\bar{\vect{u}}_{n+1}+\bar{\vect{f}}}
\end{equation}

or 

\begin{equation}
  \vect{\alpha}^{p}_{n} =\inverse{\matr{A}}\pbrac{\matr{M}\ddot{\bar{\vect{u}}}_{n+1}+
    \matr{C}\dot{\bar{\vect{u}}}_{n+1}+\matr{K}\bar{\vect{u}}_{n+1}+\bar{\vect{f}}}
\end{equation}

If $\fnof{\vect{g}}{\vect{u}}$ is not $\vect{0}$ then \eqnref{eqn:transient} is nonlinear in $\vect{\alpha}^{p}_{n}$. To solve this
equation we use Newton's method. i.e.

\begin{equation}
  \begin{split}
    \text{1.  } & \fnof{\matr{J}}{\vect{\alpha}^{p}_{n(i)}}.\delta
    \vect{\alpha}^{p}_{n(i)} = 
    -\fnof{\vect{\psi}}{\vect{\alpha}^{p}_{n(i)}} \\
    \text{2.  } & \vect{\alpha}^{p}_{n(i+1)}=\vect{\alpha}^{p}_{n(i+1)}+\delta
    \vect{\alpha}^{p}_{n(i)}
  \end{split}
\end{equation}

where $\fnof{\matr{J}}{\vect{\alpha}^{p}_{n}}$ is the Jacobian and is given by

\begin{equation}
  \fnof{\matr{J}}{\vect{\alpha}^{p}_{n}}=\dfrac{\theta_{p-2}{\Delta t}^{p-2}}{\factorial{p-2}}\matr{M}+\dfrac{\theta_{p-1}{\Delta
      t}^{p-1}}{\factorial{p-1}}\matr{C}+\dfrac{\theta_{p}{\Delta
      t}^{p}}{p!}\matr{K}+\dfrac{\theta_{p}{\Delta t}^{p}}{p!}
  \delby{\fnof{\vect{g}}{\vect{\alpha}^{p}_{n}}}{\vect{\alpha}^{p}_{n}}
\end{equation}

or

\begin{equation}
  \fnof{\matr{J}}{\vect{\alpha}^{p}_{n}}=\matr{A}+\dfrac{\theta_{p}{\Delta t}^{p}}{p!}\delby{\fnof{\vect{g}}{\vect{\alpha}^{p}_{n}}}{\vect{\alpha}^{p}_{n}}
\end{equation}

Once $\vect{\alpha}^{p}_{n}$ has been obtained the values at the next time
step can be obtained from

\begin{equation}
  \begin{split}
    \vect{u}_{n+1} &= \vect{u}_{n}+\Delta t
    \dot{\vect{u}}_{n}+\cdots+\dfrac{{\Delta t}^{p}}{p!}\vect{\alpha}^{p}_{n}=\hat{\vect{u}}_{n+1}+\dfrac{{\Delta t}^{p}}{p!}\vect{\alpha}^{p}_{n}\\
    \dot{\vect{u}}_{n+1} &= \dot{\vect{u}}_{n}+\Delta t
    \ddot{\vect{u}}_{n}+\cdots+\dfrac{{\Delta
        t}^{p-1}}{\factorial{p-1}}\vect{\alpha}^{p}_{n}=\dot{\hat{\vect{u}}}_{n+1}+\dfrac{{\Delta
        t}^{p-1}}{\factorial{p-1}}\vect{\alpha}^{p}_{n} \\
    &\vdots \\
    \symover{p-1}{\vect{u}}_{n+1} &= \symover{p-1}{\vect{u}}_{n}+\Delta t\vect{\alpha}^{p}_{n}
  \end{split}
\end{equation}


\subsubsection{Special SN11 case, p=1}

For this special case, the mean predicited values are given by
\begin{equation}
   \bar{\vect{u}}_{n+1} = \vect{u}_{n}
\end{equation}

The amplification matrix is given by
\begin{equation}
  \matr{A}=\matr{C}+\theta_{1}\Delta t \matr{K}
\end{equation}

The right hand side vector is given by
\begin{equation}
  \vect{b}=\matr{K}\bar{\vect{u}}_{n+1}+\bar{\vect{f}}
\end{equation}

The nonlinear function is given by
\begin{equation}
  \fnof{\vect{\psi}}{\vect{\alpha}^{1}_{n}}=\matr{A}\vect{\alpha}^{1}_{n}+\fnof{\vect{g}}{\bar{\vect{u}}_{n+1}+ \theta_{1}\Delta
    t\vect{\alpha}^{1}_{n}}+\vect{b}=\vect{0}
\end{equation}

The Jacobian matrix is given by
\begin{equation}
  \fnof{\matr{J}}{\vect{\alpha}^{1}_{n}}=\matr{A}+\theta_{1}\Delta t \delby{\fnof{\vect{g}}{\vect{\alpha}^{1}_{n}}}{\vect{\alpha}^{1}_{n}}
\end{equation}

And the time step update is given by
\begin{equation}
    \vect{u}_{n+1} = \vect{u}_{n}+\Delta t\vect{\alpha}^{1}_{n}
\end{equation}

